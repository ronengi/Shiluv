\selectlanguage{hebrew}

\section{שימוש  בגאוגברה}

\subsection*{ערוגת הורדים}
השטח של ערוגת הפרחים נשאר קבוע.

%%%%%%%%%%%%%%%%%%%%%%%%%%%%%%%%%%%%%%%%%%%%%%%%%%%%%%%%%%%%%%%%%%%%%%%%%%%%%%

\subsubsection*{א. יישומון אינטראקטיבי בגאוגברה}
להלן %
\hypersetup{urlcolor=blue}%
\label{https://www.geogebra.org/classic/k7ybx9jt}%
\href{https://www.geogebra.org/classic/k7ybx9jt}{קישור ליישומון גאוגברה}%
. יישומון זה מאפשר לראות מיידית שהזזת הקטע %
$ g $ %
כל עוד הוא מאונך לקטע %
$ AB $%
, כמו גם הזזת בסיס המשולש, לא משנים את שטח המשולש.
%\\[20pt]

%%%%%%%%%%%%%%%%%%%%%%%%%%%%%%%%%%%%%%%%%%%%%%%%%%%%%%%%%%%%%%%%%%%%%%%%%%%%%%

\subsubsection*{ב. פתרון בדרך פורמלית}
מצאתי שתי הוכחות (שתיהן למעשה הצגות שונות של אותה הוכחה). שתיהן מתייחסות לשרטוט הבא:
\\[10pt]
\hspace*{50pt}
\includegraphics[width=300pt]{/archive/Achva/תשף/א/05. שילוב מחשב בהוראת המתמטיקה/13 עבודה מסכמת/ronengi - shiluv 2.eps}

%\selectlanguage{hebrew}
\subsubsection*{ב.1. שימוש בטריגונומטריה}
אשתמש בנוסחת שטח משולש: מחצית מכפלת שתי צלעות בסינוס הזווית בינהן.
\\[-10pt]
\hspace*{90pt}
\parbox[t][130pt][t]{370pt}{
\begin{align}
S &= S_1 + S_2 = \\
&= [ \tfrac{1}{2} \cdot a \cdot g \cdot \sin(\alpha) ] + [ \tfrac{1}{2} \cdot b \cdot g \cdot \sin(\beta) ] = \\
&= \tfrac{1}{2} \cdot g \cdot [ a \cdot \sin(\alpha) + b \cdot \sin(\beta) ] = \\
&= \tfrac{1}{2} \cdot g \cdot [ a \cdot \sin(\beta) + b \cdot \sin(\beta) ] = \\
&= \tfrac{1}{2} \cdot g \cdot ( a + b ) \cdot \sin(\beta) = \\
&= \tfrac{1}{2} \cdot g \cdot ( a + b ) \cdot \tfrac{AB}{(a + b)} = \\
&= \tfrac{1}{2} \cdot g \cdot AB
\end{align}
}
\\[10pt]
(1), (2) שטח המשולש שווה לסכום שני חלקיו \\
(3) הוצאנו $\tfrac{1}{2} \cdot g$ גורם משותף \\
(4) $\alpha, \beta$ זוויות צמודות
$ \sin(\alpha) = \underset{\scriptscriptstyle{= \; \sin(180\degree - \alpha)}}{\sin(\beta)} \qquad \Leftarrow \qquad \beta = 180\degree - \alpha \qquad \Leftarrow \qquad $\\[-7pt]
(5) הוצאת $\sin(\beta)$ גורם משותף\\[2pt]
(6) נתבונן במשולש 3: $\sin(\beta) = \tfrac{AB}{(a + b)}$\\[2pt]
(7) צמצמנו מונה ומכנה ב: $(a + b)$
%\\
%$\bigtriangleup$
 
%$\square \qedsymbol \blacksquare $
%\selectlanguage{english}
%\selectlanguage{hebrew}
\subsubsection*{ב.2. חיבור שטחי משולשים}
אשתמש בנוסחת שטח משולש: מחצית מכפלת הצלע בגובה לצלע.
\\[-10pt]
\hspace*{100pt}
\parbox[t][80pt][t]{350pt}{
\begin{align}
S &= S_1 + S_2 = \\
&= [ \tfrac{1}{2} \cdot g \cdot AE ] + [ \tfrac{1}{2} \cdot g \cdot EB ] = \\
&= \tfrac{1}{2} \cdot g \cdot (AE + EB) = \\
&= \tfrac{1}{2} \cdot g \cdot AB
\end{align}
}
\\[10pt]
(8) שטח המשולש שווה לסכום שני חלקיו \\
(9) מכיוון ש- $g$ ניצב ל- $AB$ והגינה כולה היא מלבן, הגבהים לצלעות שווים לקטעים $AE, EB$ \\
(10) הוצאנו $\tfrac{1}{2} \cdot g$ גורם משותף \\
(11) סכום הקטעים $AE + EB$ הוא בדיוק $AB$
