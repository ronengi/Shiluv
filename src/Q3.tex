\selectlanguage{hebrew}

\section{פרויקטי ההדגמה של וולפרם}

\subsection*{}

\subsubsection*{יישומון א'}
יישומון א' מדגים כיצד למדוד במדויק ליטר אחד של מים באמצעות שני כלים, קטן וגדול, בנפחים של ליטרים %
שלמים - ללא שנתות. ההדגמה מתבצעת באמצעות ציור של מצב המים בכל כלי וגם שרטוט המצב והצעדים %
במערכת צירים מקבילית (בתיאור היישומון היא מוגדרת כמערכת צירים משולשת).%

היישומון מראה במדויק צעד צעד את האלגוריתם לפתרון. האלגוריתם נקבע אוטומטית בשיטה מכנית קבועה: %
מתחילים כאשר הכלי הגדול מלא והקטן ריק: זו הפינה הימנית התחתונה של המקבילית. הצעד הבא הוא %
לשפוך מהכלי הגדול לקטן ולמלא אותו. מכאן ואילך כל צעד נקבע כקפיצה של כדור שנע במסלול מדופן %
לדופן של המקבילית. היישומון מתעלם מהמטרה המוצהרת של ליטר מים והמסלול ממשיך עד שסיים %
לעבור בכל אחת מהנקודות על דפנות המקבילית.
\\
\hspace*{469pt}
\\
החסרונות שמצאתי:
\begin{enumerate}[=dsfasd, itemsep=-2pt, itemindent=2em] %, listparindent=0pt]
\item \parbox[t][2.2em][t]{420pt}{היישומון לא מראה את הדרך הקצרה והיעילה ביותר. האלגוריתם הוא מכני לחלוטין ונעדרים ממנו  הדמיון והיצירתיות של חשיבה אנושית.}
\item היישומון לא מודיע כאשר הושגה המטרה. הוא פשוט ממשיך עד שמוצו כל צירופי אפשרויות המילוי של הכלים.
\item \parbox[t][2.2em][t]{420pt}{היישומון לא מתריע כאשר אין אפשרות להגיע לפתרון הרצוי, וזהו המצב כאשר נפח הכלי הגדול הוא כפולה של נפח הכלי הקטן (ונפח הכלי הקטן גדול מליטר אחד).}
\item \parbox[t][2.2em][t]{420pt}{המטרה המוצהרת בתיאור היישומון היא להגיע לליטר אחד. למעשה אין מגבלה עקרונית להגיע לכל מטרה אחרת, ללא שינוי של התוכנית.}
\end{enumerate}

לדוגמה: בכלים בנפח 5, ו- 3 ליטרים לאלגוריתם של היישומון נדרשו 7 צעדים. דרך קצרה יותר תהיה להתחיל %
דוקא עם המיכל הקטן מלא (1), לרוקן אותו לגדול (2), לחזור ולמלא אותו (3), ולרוקן ממנו 2 ליטרים עד שהגדול %
מלא (4). כעת נותר ליטר אחד במיכל הקטן, וזאת ב-4 צעדים.

ואם נבחר כלים בנפחים 6, ו-3 ליטר, האלגוריתם ממשיך כאמור עד תומו ללא אינדיקציה שמשהו שונה קורה כאן. %
וזו דוקא מסקנה חשובה, במסגרת הבעיה הזו.

\subsubsection*{יישומון ב'}
יישומון ב' משתמש באסטרטגיה שונה: פתרון הכללי שמצביע על הרבה פתרונות ספציפיים.
\\
הפתרון הכללי לבעיה ניתן באמצעות משוואה דיופנטית (שפתרונותיה הם מספרים שלמים בלבד), וממנו ניתן למצוא %
את הפתרון המסוים היעיל ביותר.
\\
דבר נוסף שהפתרון הכללי מאפשר הוא להוכיח כי במצבים מסוימים פתרון אינו  אפשרי כלל.
\\
אפשר לבחור את נפחי המיכלים ואת נפח היעד הרצוי. %
על מנת לתרגם את הפתרון שנותן היישומון לאלגוריתם של ממש, יש צורך בטיפה מחשבה ויצירתיות או בהיכרות %
עם  בעיות מסוג זה.
\\\\
לדוגמה: מיכלים בגודל 3, ו-5 ליטרים, היעד הוא ליטר אחד.
\\
התוכנה משרטטת מערכת צירים קרטזית וגרף של %
המשוואה: %
$3x+5y=1$ %
. על הגרף מודגשות הנקודות ששיעוריהן %
$(x ,y)$ %
הם מספרים שלמים. כל נקודה כזו מייצגת  דרך פעולה אפשרית, ובוחרים את זו שבה הסכום %
$ |x| + |y| $ %
הוא הקטן ביותר. במקרה שלנו זו % 
$(-1, 2)$.
\\
המשמעות היא: שני מילויים של מיכל 3 ליטר וריקון אחד של מיכל 5 ליטר % 
( $6_\ell - 5_\ell = 1_\ell$ ).
\\
בעיבוד לאלגוריתם פעולה: %
מילוי של מיכל 3 ליטר (1), ריקון אל מיכל 5 ליטר (2), מילוי שוב של מיכל 3 ליטר (3), ריקון 2 ליטר מתוכו למיכל %
5 ליטר שמתמלא לגמרי (4). נשארנו עם ליטר אחד במיכל 3 ליטר (בדיוק כמו הדוגמה הקצרה יותר מהאלגוריתם %
של יישומון א').
\\
יש לשים לב שמילאנו פעמיים 3 ליטר, אבל ריקון 5 ליטר התבצע בשני שלבים, ולא סתם שפכנו את המים אלא השתמשנו בהם %
בכלי הגדול על מנת למדוד במדויק 5 ליטרים.
\\
היישומון מאפשר להציג את d, שהוא סך כל הצעדים, על מערכת הצירים, והוא מופיע כריבוע נטוי העובר בכל %
הנקודות שסכום הערכים המוחלטים של שיעוריהן הוא d. בחירת d שהולך וגדל מאפשרת לראות בבירור מהו %
הפתרון היעיל ביותר.
\pagebreak
\\

החסרונות שמצאתי:
\begin{enumerate}[itemsep=-2pt, itemindent=2em] %, listparindent=0pt]
\item \parbox[t][2.2em][t]{424pt}{האלגוריתם מורכב כמעט תמיד ממספר הגדול ממש מ- %
$ d $. % 
למשל בדוגמה שנתתי: ריקון 5 ליטרים מתבצע בשני צעדים נפרדים, ולכן מספר הצעדים הוא 4, ולא 3.}
\item \parbox[t][1.2em][t]{424pt}{האלגוריתם המדויק והיעיל ביותר מצריך מחשבה נוספת.}
\end{enumerate}
בתיאור היישומון מנוסח  כלל האומר כי הבעיה לא ניתנת לפתרון אם המחלק המשותף הגדול ביותר של נפחי %
שני המיכלים אינו מחלק גם את נפח המטרה המבוקש. למשל: בעזרת מיכלים בנפחים 3 ו-6 ליטרים, לא ניתן %
למדוד נפח של 4 ליטרים (3 לא מחלק את 4). לכן אין נקודות מודגשות בגרף. %

ליישומון  יש תצוגה נוספת: שני סרגלים צמודים, עליון ותחתון, בהם כל שנתה מסמלת ליטר אחד, ובכל סרגל %
ממוספרות רק השנתות שהן כפולה של נפח המיכל המתאים. הסרגלים זזים אחד ביחס לשני בהתאם ל- % 
$ d $ %
הנבחר: קודם כדאי לבחור %
$ d $ %
שלילי כלשהו, ואז מחפשים מקומות בהם השנתות הממוספרות של שני הסרגלים %
מתלכדות ביניהן. מחברים את המספרים האלו ומקבלים את  %
$ d $. %
הפעם בוחרים את % 
$ d $ %
חיובי בהתאם ומתחילים מהאפס של הסרגל העליון ימינה להוסיף או להחסיר שנתות כמספר שהתקבל עבור כל סרגל כפול נפח %
המיכל (לקפוץ ימינה או שמאלה 3 או 5 שנתות כל פעם במקרה שלנו). בסוף התהליך נותר הנפח המבוקש %
בסרגל העליון, מסומן באדום.

תצוגה זו לא מוסברת מספיק ולא טריויאלית להבנה וגם קטנה מאד, אפשר להגדיל אבל בכל בחירה נוספת היא %
קופצת בחזרה לגודל הממוזער. זו נראית כמו התחלה של משהו מעניין אבל עדיין לא מוגמר.

%\hspace*{300pt}