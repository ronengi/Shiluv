\selectlanguage{hebrew}

\section{שימוש בגליונות אלקטרוניים}

\subsection*{המלך אדגר}
\subsubsection*{א.}
נסמן את מספר מטילי הזהב שקיבלו ארבעת הבנים לפי מספר הבן: $x_1, x_2, x_3, x_4$.\\עלינו להתחשב במגבלות הבאות:
\\[3pt]
\hspace*{80pt}
%\fbox{
$
\left
\{
\begin{tabular}{ r l }
\text{סך הכל כולם יקבלו 1000 מטילי זהב}
\tiny \dotfill
&
$x_1 + x_2 + x_3 + x_4 = 1000$
\\
\text{הבן הבכור יקבל פי שניים מהבן השני}
\tiny \dotfill
&
$x_1 = 2 \cdot x_2$
\\
\text{הבן השלישי יקבל יותר משני הראשונים יחד}
\tiny \dotfill
&
$x_3 > x_1 + x_2$
\\
\text{הבן הרביעי יקבל פחות מהבן השני}
\tiny \dotfill
&
$x_4 < x_2$
\\
\end{tabular}
\right.
$
%}
\\[10pt]
אנו רוצים שהבן הרביעי יקבל מספר מקסימלי של מטילי זהב. היות והוא חייב להיות קטן ממש ממספר המטילים של הבן השני, המקסימום האפשרי הוא: $x_4 = x_2 - 1$ (בהנחה שלא ניתן לחלק מטיל שלם למספר חלקים).\\
על כן, עלינו לשאוף לכך שהבן השני יקבל גם הוא מספר מקסימלי של מטילים.\\
בנוסף לכך, מכיוון שהבן השלישי מקבל מספר הגדול ממש מסכום שני הראשונים,\ נשתדל שהבן השלישי יקבל מספר מינימלי של מטילי זהב מעבר לסכומם. המינימום האפשרי הוא: $x_3 = (x_1 + x_2) + 1$.\\
ונקבל את מערכת המשוואות הבאה:
\\[3pt]
\hspace*{80pt}
%\fbox{
$
\left
\{
\begin{tabular}{ r l }
\text{סך הכל כולם יקבלו 1000 מטילי זהב}
\tiny \dotfill
&
$x_1 + x_2 + x_3 + x_4 = 1000$
\\
\text{הבן הבכור יקבל פי שניים מהבן השני}
\tiny \dotfill
&
$x_1 = 2 \cdot x_2$
\\
\text{הבן השלישי יקבל יותר משני הראשונים יחד}
\tiny \dotfill
&
$x_3 = x_1 + x_2 + 1$
\\
\text{הבן הרביעי יקבל פחות מהבן השני}
\tiny \dotfill
&
$x_4 = x_2 - 1$
\\
\end{tabular}
\right.
$
%}
\\[10pt]
\begin{tabular}{ r l }
נציב את:
&
$x_1 = 2 \cdot x_2$
\\
ב:
&
$x_3 = x_1 + x_2 + 1
\\[3pt]
ונקבל:
&
$x_3 = 2 \cdot x_2 + x_2 + 1 = $
\\
&
$ = 3 \cdot x_2 + 1 $ \hspace{10pt}
\end{tabular}
\\[10pt]
כעת יש לנו את ייצוג כמויות המטילים שקיבל כל בן באמצעות $ x_2 $ בלבד.
\\
ומשוואת הסכום המתקבלת היא:
\\
\hspace*{195pt}
\LRE{\parbox{200pt}{
$ (2 x_2) + (x_2) + (3 x_2 + 1) + (x_2 - 1) = 1000 $
\\[2pt]
$ 7 x_2 + 1 - 1 = 1000 $
\\[2pt]
$ x_2 = \tfrac{1000}{7} = 142 \tfrac{6}{7} $
}}
\\[10pt]
המשמעות היא, שהמקסימום שיוכל לקבל הבן השני הם $ 142 $ מטילים שלמים, מעבר לכך הסכום יעבור את ה- $ 1000 $ מטילים.
\\
ולכן, הבן הרביעי יוכל לקבל מספר מקסימלי של $ 141 $  מטילים שלמים (את $ 6 $ המטילים הנותרים נאלץ לתת לבן השלישי\ldots).
